\documentclass{article}
\usepackage{mathtools}
\usepackage[linesnumbered,ruled,vlined]{algorithm2e}


\renewcommand{\listfigurename}{List of Algorithmes}

\begin{document}
    
    \begin{algorithm}
        \caption{Déterminier  le caractère parfait ou non d'un nombre strictement positif}
        \SetAlgoVlined
        \SetKwInOut{Input}{input}
        \SetKwInOut{Output}{output}
        \Input{N un entier...}
        \Output{nombre\textunderscore parfait un boolean}   
        \tcc{initialisation des deux variable sommediviseur et diviseur de deux varaible entiére}
        
        somme\textunderscore diviseur $\leftarrow 0$
        
        diviseur $\leftarrow 1$
        
          \tcc{calculer la some de diviseur et comparer cetter somme avec le nombre N}
          
        \While{diviseur $< N$}{
            \If{N mod diviseur = 0}{
                somme\textunderscore diviseur $\leftarrow somme\textunderscore diviseur + diviseur$
                
                diviseur $\leftarrow diviseur + 1$
            }
        
        }
        nombre\textunderscore parfait $\leftarrow (N = somme\textunderscore diviseur )$ $ \wedge (N  \neq 0) $
        
        \tcc{retourner le résultat}

        \KwRet{nombre ...}
        
        \label{alg:PoEG}        
        
    \end{algorithm}
    
        \listofalgorithms
        
    
    
\end{document}

